{
\psset{linearc=0}
{\Large How to read this chart}
\begin{itemize}

\item This chart is organized in octaves (frequency doubling/halving) starting at 1Hz and going higher (2,4,8, etc) and lower (1/2, 1/4, etc). The octave is a natural way to represent frequency.

\item Frequency increases in the upward direction. and wrap around from far right to far left.

%\item This chart was also designed so that the spectrum section could be cut our and wrapped around a standard 3 inch diameter by 36 inches long mailing tube to represent a continuous frequency from the bottom to the top and beyond.

\item There is no limit to either end of this chart, however, due to limited space, only the ``known" items have been shown here. A frequency of 0Hz is the lowest possible frequency but the method of depicting octaves used here does not allow for ever reaching 0Hz, only approaching it. Also, by the definition of frequency (Cycles per second), there is no such thing as negative frequency.

\item Values labels: \psframebox[framesep=1pt,fillstyle=solid,fillcolor=Black]{\textcolor{FColor}{Frequency}} in Hertz,%
\psframebox[framesep=1pt,fillstyle=solid,fillcolor=Black]{\textcolor{WColor}{Wavelength}} in meters,%
\psframebox[framesep=1pt,fillstyle=solid,fillcolor=Black]{\textcolor{EColor}{Energy}} in electronVolts.

\end{itemize}
}
