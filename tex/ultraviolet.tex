{\Large {\bfseries U}ltra{\bfseries v}iolet Light ({\bfseries UV})}
\begin{itemize}
\item UV light is beyond the range of human vision. A bumblebee can see light in the UVA range which helps them identify certain flowers.

% \item Physicists have divided ultraviolet light ranges into Vacuum Ultraviolet (VUV), Extreme Ultraviolet (EUV), Far Ultraviolet (FUV), Medium Ultraviolet (MUV), and Near Ultraviolet (NUV).

% \item UV-A, UV-B and UV-C were introduced in the 1930's by the Commission Internationale de l'\'{E}clairage (CIE, International Commission on Illumination) for photobiological spectral bands.

\item Short-term UV-A exposure causes sun-tanning which helps to protect against sunburn. Exposure to UV-B is beneficial to humans by helping the skin produce vitamin D. Excessive UV exposure causes skin damage. UV-C is harmful to humans but is used as a germicide.

% \item The CIE originally divided UVA and UVB at 315nm, later some photo-dermatologists divided it at 320nm.

% \item UVA is subdivided into UVA1 and UVA2 for DNA altering effects at 340nm.

\item The Sun produces a wide range of wavelengths including all the UV light, however, UVB is partially filtered by the ozone layer and UVC is totally filtered out by the earth's atmosphere.

\end{itemize}




% From: http://www.ping.at/cie/publ/abst/134-99.html
% Press Release:
% CIE Collection in Photobiology and Photochemistry, 1999
% CIE 134-1999 ISBN 3 900 734 94 1
% This volume contains short Technical Reports prepared by various Technical Committees within CIE Division 6.

% 134/1: TC 6-26 report: Standardization of the Terms UV-A1, UV-A2 and UV-B
% The terms UV-A, UV-B and UV-C were introduced in the 1930's by CIE Committee 41 on Ultraviolet Radiation as a short-hand notation for photobiological spectral bands. It was never intended that the bands were exclusive for different effects. The bands have been in widespread use in different medical fields and scientific research. UV-A and UV-B were divided at 315 nm by the CIE. In recent decades, some photo-dermatologists and others have used different dividing lines such as 320 nm without recognizing the importance of maintaining an international standardized terminology. Because the terminology is used in many fields, this report recommends that the 315 nm division between UV-A and UV-B be maintained. However, recent research has clearly shown a difference in the photobiological interaction of long and short wavelength UV-A radiation with DNA. This led to a further division of UV-A into UV-A1 and UV-A2 with a dividing line at approximately 340 nm. While this division may be of value, the committee does not recommend officially to split UV-A into these two sub-bands at this time. Further research may justify a dividing line different from 340 nm in the future.
